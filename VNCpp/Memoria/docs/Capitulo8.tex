\section{Conclusiones}
Como conclusión pensamos que se han alcanzado las metas propuestas en el apartado 1.3. Todo el código ha sido licenciado bajo una licencia libre (GPL versión 3), por tanto el objetivo de crear una aplicación de software libre se ha cumplido sin mayor problema. En cuanto al objetivo más importante, la construcción de la aplicación de escritorio remoto en Android utilizando código nativo, también se ha cumplido, ya que no sólo se ha construido una aplicación con muchas funcionalidades, sino que es una aplicación eficiente, capaz de competir con todas las demás aplicaciones de escritorio remoto para Android que existen en este momento. Cabe destacar una vez más que la gran diferencia y la razón por la que este proyecto es innovador, es en que ninguna de las demás aplicaciones utilizan código nativo, al menos no las de software libre. La imposibilidad de saber como están construidas las aplicaciones privativas nos impide afirmar con total rotundidad que no existe una aplicación de escritorio remoto para Android que utilice el NDK. No obstante, al no existir una de software libre, la innovación sigue estando ahí, ya que hemos trabajado sobre algo que no existe en ningún lado y con unos resultados considerablemente buenos.

\section{Trabajo Futuro}

Como posible trabajo futuro se podría añadir soporte para todos los idiomas, en dos sentidos. Por una lado la traducción de la interfaz a diveros idiomas y, por otro, soporte en el teclado para más idiomas.\\

Otra cosa que se podría investigar más es la aceleración por hardware, de esta manera se podría conseguir una mayor velocidad a la hora de trabajar las imágenes y con ello se obtendría una mejora del rendimiento global considerable.\\

También pensar en portar la aplicación a las demás plataformas móviles como Firefox OS. 
