
Nowadays there are a lot of VNC based applications for Android. Nonetheless, none of these applications are Free Software and are implemented using the NDK. The point of this project is to use the NDK potential to make a VNC based application and try to obtain a better performance than the other applications implemented using only the SDK and all the source code will be Free Software.\\

The protocol to comunicate with a VNC server is called RFB. To implementing this protocol in this project has been used the libVNCServer library. This library facilitates the control of RFB. Without libVNCServer it would be more difficult to make our own library to manage the protocol. JNI is used as the framework to comunicate the native code with Java. The technologies, as well as the process used to built this software is going to be explained along this report. 
\\ \mbox{ } \\
\textit{\textbf{Keywords}}

VNC, Android, C++, C, NDK, SDK, JNI, RFB.
